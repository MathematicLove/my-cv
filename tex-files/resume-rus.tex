\documentclass[10pt,a4paper]{article}
\usepackage[utf8]{inputenc}
\usepackage[T2A]{fontenc}
\usepackage[english, russian]{babel}
\usepackage{xifthen}
\usepackage[default]{raleway}
\usepackage{hyperref}
\renewcommand*\familydefault{\sfdefault}
\usepackage{xcolor}
\usepackage{moresize}
\usepackage[a4paper]{geometry}
\geometry{top=1.75cm, bottom=-.6cm, left=1.5cm, right=1.5cm}
\usepackage{fancyhdr}
\pagestyle{fancy}
\setlength{\headheight}{-5pt}

\lhead{}
\chead{\small{
  \textbf{Салимли Айзек $\cdot$
  Java-разработчик $\cdot$
  Санкт-Петербург, Россия $\cdot$
  \textcolor{sectcol}{\textbf{\href
  {mailto:thisisnauchno@gmail.com}
  {thisisnauchno@gmail.com}}} $\cdot$
  +7 921 945 67 03
}}}
\rhead{}

\setlength{\parindent}{0mm}

\usepackage{multicol}
\usepackage{multirow}
\usepackage{array}
\newcolumntype{x}[1]{>{\raggedleft\hspace{0pt}}p{#1}}
\usepackage{graphicx}
\usepackage{wrapfig}
\usepackage{float}
\usepackage{tikz}
\usetikzlibrary{shapes, backgrounds,mindmap, trees}
\usepackage{color}
\definecolor{sectcol}{RGB}{90,90,120}
\definecolor{bgcol}{RGB}{110,110,110}
\definecolor{softcol}{RGB}{225,225,225}

\renewcommand{\headrulewidth}{0pt}

\renewcommand{\footrulewidth}{0pt}

\renewcommand{\thepage}{}

\renewcommand{\thesection}{}


\newcommand{\tzlarrow}{(0,0) -- (0.2,0) -- (0.3,0.2) -- (0.2,0.4) -- (0,0.4) -- (0.1,0.2) -- cycle;}

\newcommand{\larrow}[1]{
  \begin{tikzpicture}[scale=0.58]
    \filldraw[fill=#1!100,draw=#1!100!black]  \tzlarrow
  \end{tikzpicture}
}

\newcommand{\tzrarrow}{ (0,0.2) -- (0.1,0) -- (0.3,0) -- (0.2,0.2) -- (0.3,0.4) -- (0.1,0.4) -- cycle;}

\newcommand{\rarrow}{
  \begin{tikzpicture}[scale=0.7]
    \filldraw[fill=sectcol!100,draw=sectcol!100!black] \tzrarrow
  \end{tikzpicture}
}

\newcommand{\cvsection}[1]{
  \vspace{10pt}
  \colorbox{sectcol}{\mystrut \makebox[1\linewidth][l]{
    \larrow{bgcol} \hspace{-8pt} \larrow{bgcol} \hspace{-8pt}
    \larrow{bgcol}\textcolor{white}{\textbf{#1}}\hspace{4pt}
  }}\\
}

\newcommand{\metasection}[2]{
  \begin{tabular*}{1\textwidth}{p{2.4cm} p{11cm}}
    \larrow{bgcol} \normalsize{\textcolor{sectcol}{#1}}&#2\\[10pt]
  \end{tabular*}
}

\newcommand{\cvevent}[3]{
  \begin{tabular*}{1\textwidth}{p{2.5cm} p{10.5cm} x{4.0cm}}
    \textcolor{bgcol}{#1} & \textbf{#2} & \vspace{2.5pt}\textcolor{sectcol}{#3}
  \end{tabular*}
  \vspace{-10pt}
  \textcolor{softcol}{\hrule}
  \vspace{10pt}
}

\newcommand{\cvdetail}[1]{
  \begin{tabular*}{1\textwidth}{p{2.5cm} p{14.5cm}}
    & \larrow{bgcol}  #1\\[3pt]
  \end{tabular*}
}

\newcommand{\mystrut}{\rule[-.3\baselineskip]{0pt}{\baselineskip}}

\title{резюме}
\begin{document}

\pagestyle{fancy}

\vspace{-20pt}

\hspace{-0.25\linewidth}\colorbox{bgcol}{
  \makebox[1.5\linewidth][c]{
    \HUGE{\textcolor{white}{\textsc{Салимли Айзек}}}
    \textcolor{sectcol}{\rule[-1mm]{1mm}{0.9cm}}
    \HUGE{\textcolor{white}{\textsc{Резюме}}}
  }
}

\begin{figure}[H]
\begin{flushright}
  \includegraphics[width=0.2\linewidth]{img/ME.jpeg}
\end{flushright}
\end{figure}

\vspace{-115pt}

\metasection{Статус:}{Студент бакалавриата (год выпуска 2026), СПбПУ, ИКНК, Математика и компьютерные науки}
\\
\metasection{Языки:}{Русский (носитель), Английский (B2)}
\\
\metasection{Навыки:}{Java (core), Spring Framework, Spring MVC, Hibernate, Spring JPA, Spring JDBC, Concurrency, Maven, Gradle}
\\
\metasection{Навыки DS/ML:}{Python, Numpy, Pandas, YOLO, Scikit-learn, Scikit-image, PyTorch, Matplotlib, Seaborn}
\\
\metasection{Прочие навыки:}{Git, PostgreSQL, MySQL, MongoDB, Redis, RabbitMQ, Docker, k8s, REST API, MVC, TCP/IP, Алгоритмы и структуры данных, Машинное обучение, Прогнозирование, Методы оптимизации, Теория вероятностей, Мат. статистика, Квантовые вычисления, Дискретная математика, ER-диаграммы, Use cases, BPM, BPMN}
\\
\metasection{Другие Я.П.:}{\textcolor{magenta}{Haskell}, \textcolor{olive}{C++}, \textcolor{blue}{Python}, \textcolor{purple}{R}}
\\
\metasection{Интересы:}{Математика в компьютерных науках, Путешествие, Футбол, Музыка}

\cvsection{О себе}
\\
Junior Java-разработчик с сильной математической подготовкой, завершающий обучение по программе бакалавриата "Математика и компьютерные науки" в Санкт-Петербургском политехническом университете Петра Великого (СПбПУ). Практический опыт работы с Spring Framework, Hibernate и разработкой REST API, дополненный знаниями в области машинного обучения, алгоритмов и систем баз данных. Доказанная способность решать задачи оптимизации (VRP/DVRP) и внедрять решения на основе ИИ с использованием LLM-технологий (LangChain, OpenAI API). Стремлюсь применять математическое моделирование для решения практических задач в разработке ПО и анализе данных. Ищу возможности для профессионального роста в качестве Backend/Java-разработчика и участия в инновационных проектах.

\cvsection{Опыт работы}

\cvevent{'08/24 - 01/26}{ML-инженер и Back-end разработчик}{ \href{https://thebloomsbridge.io}
      {\textcolor{red}{\textbf{TheBloomsBridge}}}}
\noindent
\begin{minipage}{0.1\linewidth}
  \includegraphics[width=\linewidth]{img/TBB_LOGO.png}
\end{minipage}%
\begin{minipage}{0.5\linewidth}
\cvdetail{Интеграция LLM-модели, обученной на собственных данных, с использованием LangChain и OpenAI API}
\cvdetail{Многофакторный дисперсионный анализ и прогнозирование с помощью регрессионного анализа}
\cvdetail{Разработка полной пользовательской панели управления (dashboard)}
\cvdetail{Решение задач маршрутизации транспорта (VRP) и динамической маршрутизации (DVRP) для водителей доставки}
\cvdetail{Деплой приложения на Firebase}
\end{minipage}

\cvsection{Образование}
\cvevent{}{Бакалавриат, Математика и компьютерные науки}{ \href{https://www.spbstu.ru}
      {\textcolor{red}{\textbf{СПбПУ}}}}
\noindent
\begin{minipage}{0.1\linewidth}
  \includegraphics[width=\linewidth]{img/POLY_LOGO.png}
\end{minipage}%
\begin{minipage}{0.5\linewidth}
  \cvdetail{Тема диплома: Распознавание посторонних объектов на трамвайных путях в реальном времени.}
  \cvdetail{Статья: Оптимизация распределения такси на основе динамической VRP.}
  \cvdetail{Статья: Безопасность на железной дороге с использованием компьютерного зрения}
  \cvdetail{Статья: Автоматизация определения положения "вне игры" с помощью алгоритмов компьютерного зрения}
\end{minipage}

\null
\vspace*{\fill}
\hspace{-0.25\linewidth}\colorbox{bgcol}{
  \makebox[1.5\linewidth][c]{
    \mystrut \small
    \textcolor{white}{
      \href{https://github.com/MathematicLove}
      {\textcolor{red}{\textbf{https://github.com/MathematicLove}}}
    }
  }
}

\end{document}